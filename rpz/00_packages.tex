\renewcommand\labelitemi{--}

\usepackage{threeparttable}
\usepackage{algorithm}
\usepackage{algpseudocode}
\usepackage{longtable}
\usepackage{booktabs}
\usepackage{svg}
\usepackage[figure,table]{totalcount}
\usepackage{array}
\usepackage{makecell}
\usepackage{pdfpages}

\renewcommand{\listalgorithmname}{Список алгоритмов}
\floatname{algorithm}{Алгоритм}

\algrenewcommand\algorithmicwhile{\textbf{До тех пока}}
\algrenewcommand\algorithmicdo{\textbf{выполнять}}
\algrenewcommand\algorithmicrepeat{\textbf{Повторять}}
\algrenewcommand\algorithmicuntil{\textbf{Пока выполняется}}
\algrenewcommand\algorithmicend{\textbf{Конец}}
\algrenewcommand\algorithmicif{\textbf{Если}}
\algrenewcommand\algorithmicelse{\textbf{иначе}}
\algrenewcommand\algorithmicthen{\textbf{тогда}}
\algrenewcommand\algorithmicfor{\textbf{Цикл}}
\algrenewcommand\algorithmicforall{\textbf{Выполнить для всех}}
\algrenewcommand\algorithmicfunction{\textbf{Функция}}
\algrenewcommand\algorithmicprocedure{\textbf{Процедура}}
\algrenewcommand\algorithmicloop{\textbf{Зациклить}}
\algrenewcommand\algorithmicrequire{\textbf{Условия:}}
\algrenewcommand\algorithmicensure{\textbf{Обеспечивающие условия:}}
\algrenewcommand\algorithmicreturn{\textbf{Возвратить}}
\algrenewtext{EndWhile}{\textbf{Конец цикла}}
\algrenewtext{EndLoop}{\textbf{Конец зацикливания}}
\algrenewtext{EndFor}{\textbf{Конец цикла}}
\algrenewtext{EndFunction}{\textbf{Конец функции}}
\algrenewtext{EndProcedure}{\textbf{Конец процедуры}}
\algrenewtext{EndIf}{\textbf{Конец условия}}
\algrenewtext{EndFor}{\textbf{Конец цикла}}
\algrenewtext{BeginAlgorithm}{\textbf{Начало алгоритма}}
\algrenewtext{EndAlgorithm}{\textbf{Конец алгоритма}}
\algrenewtext{BeginBlock}{\textbf{Начало блока. }}
\algrenewtext{EndBlock}{\textbf{Конец блока}}
\algrenewtext{ElsIf}{\textbf{иначе если }}

\usepackage[T1,T2A]{fontenc}
\usepackage[utf8]{inputenc}
\usepackage[english,main=russian]{babel}
\usepackage{fix-cm}

\usepackage[
	left=30mm,
	right=10mm,
	top=20mm,
	bottom=20mm,
]{geometry}

\usepackage{microtype}
\sloppy

\usepackage{setspace}
\onehalfspacing

\usepackage{indentfirst}
\setlength{\parindent}{12.5mm}

\makeatletter
\renewcommand\LARGE{\@setfontsize\LARGE{22pt}{20}}
\renewcommand\Large{\@setfontsize\Large{20pt}{20}}
\renewcommand\large{\@setfontsize\large{16pt}{20}}
\makeatother
\usepackage{titlesec}
\titleformat{\chapter}[block]{\hspace{\parindent}\large\bfseries}{\thechapter}{0.5em}{\large\bfseries\raggedright}
\titleformat{name=\chapter,numberless}[block]{}{}{0pt}{\large\bfseries\centering}
\titleformat{\section}[block]{\hspace{\parindent}\large\bfseries}{\thesection}{0.5em}{\large\bfseries\raggedright}
\titleformat{\subsection}[block]{\hspace{\parindent}\large\bfseries}{\thesubsection}{0.5em}{\large\bfseries\raggedright}
\titleformat{\subsubsection}[block]{\hspace{\parindent}\large\bfseries}{\thesubsection}{0.5em}{\large\bfseries\raggedright}
\titlespacing{\chapter}{12.5mm}{-22pt}{10pt}
\titlespacing{\section}{12.5mm}{10pt}{10pt}
\titlespacing{\subsection}{12.5mm}{10pt}{10pt}
\titlespacing{\subsubsection}{12.5mm}{10pt}{10pt}
\makeatletter
\renewcommand*{\l@chapter}[2]{
\ifnum \c@tocdepth>\m@ne
    \addpenalty{-\@highpenalty}
    \vskip 1em \@plus.2em
    \@dottedtocline{0}{0pt}{1.5em}{\bfseries #1}{\bfseries #2}
\fi
}
\makeatother

\usepackage{xcolor}
\usepackage{graphicx}
\usepackage{float}
\usepackage{wrapfig}
\usepackage{tikzscale}
\usepackage{pgfplots}
\pgfplotsset{compat=newest}
\newcommand{\includeimage}[5]{%
	\ifthenelse{\equal{#2}{f}}{%
		\begin{figure}[#3]
			\center{\includegraphics[width=#4]{inc/img/#1}}%
			\caption{#5}%
			\label{img:#1}%
		\end{figure}%
	}{%
		\ifthenelse{\equal{#2}{w}}{%
			\begin{wrapfigure}{#3}{#4}
				\center{\includegraphics[width=#4]{inc/img/#1}}%
				\caption{#5}%
				\label{img:#1}%
			\end{wrapfigure}%
		}{%
			\PackageError{bmstu}{unknown image type}{}%
		}%
	}%
}

\usepackage{listings}
\usepackage{listingsutf8}
\lstset{
	inputencoding=utf8/koi8-r,
	basicstyle=\small\ttfamily,
	rulecolor=\color{black},
	escapeinside={\%*}{*)},
	breaklines=true,
	breakatwhitespace=true,
	tabsize=4,
	showstringspaces=false,
	float=h!,
	abovecaptionskip=-5pt,
}
\definecolor{numbers}{rgb}{0.5,0.5,0.5}
\definecolor{keywords}{rgb}{0.13,0.13,1}
\definecolor{comments}{rgb}{0,0.5,0}
\definecolor{strings}{rgb}{0.9,0,0}
\newcommand{\includelisting}[2]{%
	\lstinputlisting[
		frame=single,
		caption={#2},
		label={lst:#1},
	]{inc/lst/#1}%
}
\newcommand{\includelistingpretty}[3]{%
	\lstinputlisting[
		language={#2},
		keywordstyle=\color{keywords},
		stringstyle=\color{strings},
		commentstyle=\color{comments},
		frame=leftline,
		numbers=left,
		numberstyle=\footnotesize\color{numbers},
		caption={#3},
		label={lst:#1},
	]{inc/lst/#1}%
}

\usepackage{tabularx}
\usepackage{booktabs}
\usepackage[
	labelsep=endash,
	figurename=Рисунок,
	singlelinecheck=false,
]{caption}
\captionsetup[figure]{justification=centering}

\usepackage{lscape}
\usepackage{afterpage}

\usepackage{amsmath}
\usepackage{amssymb}

\usepackage[
	style=gost-numeric,
	language=auto,
	autolang=other,
	sorting=none,
]{biblatex}
\usepackage{csquotes}
\DeclareFieldFormat{urldate}{(дата обращения:\addspace\thefield{urlday}\adddot \thefield{urlmonth}\adddot \thefield{urlyear})}

\usepackage[unicode,hidelinks]{hyperref}

\usepackage{xifthen}
\usepackage{enumitem}
\usepackage{lastpage}
\usepackage{totcount}
\usepackage{assoccnt}
\newcounter{appendixchapters}
\DeclareAssociatedCounters{chapter}{appendixchapters}
\regtotcounter{appendixchapters}

\newcommand{\definition}[2]{%
	\item \noindent #1 --- #2
}

\makeatletter
\renewcommand{\tableofcontents}{%
    \chapter*{\centering \contentsname}%
    \@starttoc{toc}%
}
\makeatother

\newcommand{\maketableofcontents}{%
	\renewcommand\contentsname{СОДЕРЖАНИЕ}%
	\tableofcontents
}
\newcommand{\makebibliography}{%
	\addcontentsline{toc}{chapter}{СПИСОК ИСПОЛЬЗОВАННЫХ ИСТОЧНИКОВ}%
	\printbibliography[title=СПИСОК ИСПОЛЬЗОВАННЫХ ИСТОЧНИКОВ]%
}

\newcommand{\makeappendix}{%
	\chapter*{\centering ПРИЛОЖЕНИЕ А}%
	\addcontentsline{toc}{chapter}{ПРИЛОЖЕНИЕ А}%
}
