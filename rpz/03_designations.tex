\chapter*{СПИСОК ОБОЗНАЧЕНИЙ И СОКРАЩЕНИЙ}
\addcontentsline{toc}{chapter}{СПИСОК ОБОЗНАЧЕНИЙ И СОКРАЩЕНИЙ}

В настоящей ВКР применяют следующие сокращения и обозначения:

\begin{itemize}
    \item ЛПР
\end{itemize}

\Edit{
СПИСОК ОБОЗНАЧЕНИЙ И СОКРАЩЕНИЙ начинают со слов: "В
настоящей ВКР применяют следующие сокращения и обозначения".
При
составлении перечня для каждого обозначения приводят необходимые
сведения. Если перечень не составляется, то необходимые сведения указывают
в тексте ВКР при первом упоминании.
Перечень
сокращений,
условных
обозначений,
символов,
единиц
физических величин и определений должен располагаться столбцом без знаков
препинания в конце строки. Слева без абзацного отступа в алфавитном порядке
приводятся сокращения, условные обозначения, символы, единицы физических
величин, а справа через тире - их детальная расшифровка.}