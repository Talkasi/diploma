\chapter{Основные методы}

\section{Методы скаляризации}

В задачах многокритериального поиска путей в графах методы скаляризации
применяются для сведения исходной векторной задачи оптимизации к
последовательности однокритериальных задач поиска пути. В качестве
допустимых решений рассматриваются пути в графе, соединяющие заданные
начальную и конечную вершины. Каждому пути $p$ ставится в соответствие
вектор значений частных критериев
$f(p) = (f_1(p), \ldots, f_m(p))$,
которые определяют показатели качества данного пути. Значения частных критериев $f_i(p)$ вычисляются путём агрегирования
стоимостей рёбер, входящих в рассматриваемый путь. Скаляризация заключается
в построении скалярной целевой функции $J(f(p))$, значения которой
используются для сравнения допустимых путей в рамках однокритериальной
задачи оптимизации.

В результате применения методов скаляризации исходная многокритериальная задача поиска путей сводится к однокритериальной задаче на графе. Для её
решения могут применяться классические алгоритмы поиска кратчайшего пути в графе или их модификации.

\subsection{Метод главного критерия}

Метод предполагает выбор одного функционала \(f_i(x)\) в качестве целевой функции. Остальные требования к результату, описываемые функционалами \(f_1,\) ... \(,f_m\),
учитываются с помощью введения необходимых дополнительных ограничений~\cite{avansky2007problems}. Тогда многокритериальная задача сводится к однокритериальной, вида:
\begin{equation}
	f_1(x) \to \max_{x \in D'},
	D' \subseteq D \subseteq \mathbb{R}^n, 
	D' = \{x \in D \mid f_i(x) \geqslant t_i,\; i = 2, \ldots, m\}.
\end{equation}
где \(D'\)~---~новое допустимое множество, \(t_i\) задают минимально допустимые уровни по второстепенным критериям. 

Подход требует обоснованного выбора <<главного>> критерия из многих.

\subsection{Линейная свёртка}

Данный метод позволяет заменить векторный критерий оптимальности \(f=(f_1,\) ... \(,f_m)\) на скалярный \(J\): \(D \to R\). Все частные целевые функционалы линейно объединяются в один скаляр с помощью весов \(\alpha_i\):
\begin{equation}
	J(x) = \sum_{i = 1}^{m} \alpha_i f_i(x) \to \max_{x \in D},
\end{equation}
где \(\alpha_i \geqslant 0\), а \(\sum_{i = 1}^{m} \alpha_i = 1\).

Весовые коэффициенты \(i\) могут при этом рассматриваться как показатели относительной значимости отдельных критериальных функционалов \(f_i\)~\cite{avansky2007problems}: чем важнее \(f_i\), тем больше должно быть \(\alpha_i\). 

\subsection{Максиминная свёртка}

В данном методе, в отличие от метода линейной свёртки, на целевой функционал \(J(x)\) оказывает влияние только тот
частный критерий оптимальности, которому в данной точке \(x\) соответствует наименьшее значение функции
\(f_i(x)\): 
\begin{equation}
	J(x) = \min f_i(x) \to \max_{x \in D}.
\end{equation}

Если в случае линейной свёртки, возможны <<плохие>> значения некоторых \(f_i\) за счёт достаточно <<хороших>> значений остальных целевых функционалов, то в случае максиминного критерия производится расчёт <<на наихудший случай>>~\cite{avansky2007problems}. Так, по значению \(J(x)\) можно определить гарантированную нижнюю оценку для
всех функционалов \(f_i(x)\). Этот факт расценивается как преимущество максиминного критерия перед методом
линейной свёртки~\cite{avansky2007problems}.

Следует отметить, что методы скаляризации в общем случае не позволяют получить
всё множество Парето-оптимальных путей. В зависимости от выбранной функции
свёртки и её параметров может быть найдено отдельное Парето-оптимальное решение
или компромиссный путь, обеспечивающий приемлемое соотношение между критериями.

\section{Точные методы}

Точные методы вычисляют полное множество Парето-оптимальных путей, обеспечивая получение всех недоминируемых решений. Их основной практический предел связан с тем, что количество таких путей может расти экспоненциально даже на графах умеренного размера~\cite{paixao2007, euler2024}.

\subsection{Меточные методы}

Основная идея меточных методов состоит в последовательном 
построении и обновлении множества недоминируемых промежуточных путей, каждый из которых 
представлен меткой, содержащей вектор значений критериев. В отличие от однокритериального случая, 
где каждой вершине соответствует единственная метка, многокритериальная постановка приводит 
к необходимости хранения множества недоминируемых меток, каждая из которых описывает отдельный уникальный недоминируемый путь.

В зависимости от того, к какому элементу графа привязывается метка, различают два класса методов: вершинно-меточные и дугово-меточные. В вершинно-меточных методах метка 
ассоциируется с вершиной графа и представляет состояние пути при достижении этой вершины. 
В дугово-меточных методах метка привязывается к состоянию пути непосредственно после прохождения 
определённой дуги, а обновление метки осуществляется посредством применения дуговой функции 
расширения при добавлении очередной дуги к частичному пути.

В обоих классах методов на каждой итерации выбирается 
необработанная метка, выполняется расширение соответствующего подпути, формируются новые метки 
и проводится проверка доминирования. Метка сохраняется для дальнейшего расширения, если она 
не доминируется существующими метками для того же состояния пути, а все доминируемые метки 
удаляются. Процесс продолжается до тех пор, пока множество необработанных меток не станет пустым, 
что гарантирует построение полного множества недоминируемых путей. В зависимости от стратегии 
выбора меток выделяют метко-установочные и метко-исправляющие алгоритмы, применимые как в 
вершинной, так и в дуговой формулировке~\cite{paixao2007,euler2024}:

\begin{itemize}
    \item Метко-установочные~(label-setting): метки сканируются таким образом, что хотя бы одна метка становится финальной, означая, что наилучший путь из \(s\) в определённую вершину найден. Для этой техники требуется гарантировать, что выбранная метка может стать окончательно недоминируемой;
    \item Метко-исправляющие~(label-correcting): любая метка может обновляться до тех пор, пока не выполнится условие остановки.
\end{itemize}

\subsection{Методы ветвей и границ}

Методы ветвей и границ~(branch-and-bound) для многокритериальной оптимизации основаны на поэтапном разбиении
исходной задачи на подзадачи и оценке их границ. В многокритериальной версии метода каждая подзадача сопровождается нижней оценкой, отражающей достижимые значения
критериев, и набором найденных недоминируемых решений, используемых в качестве верхней границы.
Подзадача исключается из рассмотрения, если при учёте наложенных ограничений в ней не существует ни одного допустимого решения,
если её нижняя оценка совпадает с найденным решением или если она полностью доминируется
текущим множеством недоминируемых точек~\cite{bauss2023}.

Алгоритм многокритериального метода ветвей и границ работает итеративно. На каждом шаге выбирается подзадача для обработки
в соответствии со стратегией выбора. После этого для выбранной подзадачи вычисляется нижняя
оценка. Если на этапе вычисления обнаружено допустимое решение, оно сравнивается с текущим
множеством найденных решений и, при необходимости, обновляет его. Определяется, подлежит
ли подзадача отсечению. Если отсечение невозможно, выполняется ветвление: исходная подзадача
разбивается на две новые, каждая из которых добавляется в список необработанных подзадач.
Процесс повторяется до тех пор, пока все подзадачи не будут обработаны или отброшены.

\section{Метаэвристические методы}

Метаэвристические методы предназначены для получения приближённого множества Парето-оптимальных путей при ограниченных вычислительных ресурсах. В отличие от точных методов, они не гарантируют нахождения всего множества Парето, однако способны эффективно работать на графах большой размерности и обеспечивать разнообразие найденных решений~\cite{PangilinanJanssens2007}.

\subsection{Эволюционные алгоритмы}

Эволюционные алгоритмы представляют собой класс стохастических методов поиска, вдохновлённых идеями естественного отбора и генетики. В этих алгоритмах используется популяция решений, каждое из которых кодирует допустимую точку пространства поиска~(в рассматриваемой задаче~---~путь в графе).

Классический эволюционный алгоритм включает следующие этапы~\cite{PangilinanJanssens2007}:
\begin{enumerate}
\item генерация начальной популяции;
\item оценка качества особей;
\item отбор родительских решений;
\item применение операторов вариации~(скрещивания и мутации);
\item формирование новой популяции.
\end{enumerate}

В эволюционных алгоритмах поддерживается множество потенциальных решений, для которых на каждой итерации проводится оценка качества. Однако в многокритериальных задачах качество решения не может быть выражено одним скалярным показателем. Вместо этого рассматривается множество недоминируемых решений~\cite{PangilinanJanssens2007}. Поэтому используется отношение доминирования по Парето, а целью эволюционного алгоритма является приближение множества Парето-оптимальных путей.

Для эволюционных алгоритмов над маршрутами естественными являются следующие типы операторов:
\begin{itemize}
\item мутация: локальное изменение пути, например вставка новой вершины, замена части маршрута альтернативным подмаршрутом, удаление циклов;
\item скрещивание: обмен подмаршрутами между двумя родительскими путями при условии сохранения связности и допустимости результата.
\end{itemize}

Показано~\cite{Horoba2009}, что простой эволюционный алгоритм, применённый к многокритериальной задаче кратчайшего пути, может обладать строгими аппроксимационными гарантиями: он обеспечивает построение приближённого множества Парето за полиномиальное время относительно размера входных данных.

\subsection{Методы роя частиц}

Методы роя частиц относятся к популяционным метаэвристикам и основаны на идее коллективного поведения групп организмов, таких как стаи птиц или косяки рыб. В исходной версии алгоритм был предложен Кеннеди и Эберхартом в 1995 году~\cite{vereshchaga2023design}.

В рассматриваемой задаче каждая частица представляет собой допустимый маршрут в графе. Для частицы вычисляется вектор значений критериев, отражающий качество маршрута. Сравнение решений выполняется с помощью отношения доминирования по Парето.

Алгоритм поддерживает популяцию частиц и внешний архив недоминируемых решений. Архив служит ориентиром для поиска, поскольку в многокритериальной задаче отсутствует единственное оптимальное решение. Каждая частица также запоминает своё личное лучшее состояние.

Положение частиц обновляется через дискретные изменения маршрутов, направленные на приближение к личному лучшему решению и выбранному элементу архива. После изменения маршруты корректируются, чтобы оставаться допустимыми (например, удаляются циклы). Далее новые решения сравниваются с архивом: доминируемые исключаются, недоминируемые добавляются. Если архив становится слишком большим, из него удаляют решения из областей с высокой плотностью, чтобы сохранять равномерное представление фронта Парето.

На выходе алгоритм формирует множество недоминируемых маршрутов, которое является приближением множества Парето для многокритериальной задачи поиска пути.
