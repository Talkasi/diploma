\chapter*{\hfill{\centering  ЗАКЛЮЧЕНИЕ}\hfill}
\addcontentsline{toc}{chapter}{ЗАКЛЮЧЕНИЕ}

В работе рассмотрены группы методов решения задачи мультикритериального поиска путей в графах. 
Полученные результаты позволяют сделать вывод о том, что универсального метода, одновременно обеспечивающего высокую точность, полноту и низкую вычислительную сложность, не существует. Можно сделать вывод о том, что при выборе подхода необходимо пойти на компромисс между требуемым качеством решения и допустимыми ресурсными затратами.

Цель данной работы была выполнена. 

Для достижения поставленной цели были выполнены следующие задачи:
\begin{enumerate}[label=---]
	\item проанализирована предметная область логистики и транспортных сетей;
	\item в виде IDEF0 диаграммы формализована задача мультикритериального поиска путей в графах;
	\item рассмотрены основные методы решения задачи мультикритериального поиска путей;
	\item проведён сравнительный анализ рассмотренных методов.
\end{enumerate}
