\chapter{Анализ предметной области}

Транспортная сеть~---~совокупность транспортных линий, соединяющих узлы (города, предприятия, перекрёстки); практически представляется как граф с вершинами~---~узлами и дугами~---~линейными участками. В логистике под транспортной сетью понимают пространственную сеть, обеспечивающую движение транспорта или потока товара~\cite{shaitura2017distributed}. Существуют комплексные транспортные сети, объединяющие автодорожные, железнодорожные, воздушные и водные сети и связанную инфраструктуру.

Логистика охватывает планирование и управление потоками грузов и пассажиров, опираясь на транспортную сеть: в закупочной, производственной, распределительной и транспортной логистике оптимизируют перемещение ресурсов с учётом разных критериев, в том числе времени, стоимости и надёжности. 

\section{Основные определения}

\subsection{Задачи оптимизации}

В инженерных и прикладных задачах оптимизация рассматривается как процесс поиска такого состояния системы или конструкции, которое обеспечивает максимальное или минимальное значение заданной функции при фиксированных условиях~\cite{vereshchaga2023design}. 

Требуется найти вектор $\mathbf{x}^* = (x_1^*,\dots,x_j^*,\dots,x_n^*)^\intercal$, 
доставляющий минимум (максимум) функции $y = f(\mathbf{x})$ с заданной 
точностью $\varepsilon$, где $\mathbf{x} \in \mathbb{R}^n$.

Как правило, область допустимых значений D задаётся. Тогда задача формализуется целевой функцией в области допустимых значений D:
\begin{equation}
    f(\mathbf{x}) \to \min_{\mathbf{x} \in D}, \quad f(\mathbf{x}) \to \max_{\mathbf{x} \in D}
\end{equation}

Допустимым решением называется решение, удовлетворяющее всем ограничениям задачи, то есть принадлежащее множеству \(D\).

Многокритериальная оптимизация~(МКО)~---~это одновременная оптимизация минимум двух~(и более) конфликтующих между собой целевых функций в заданной области определения~\cite{khmelevski2024realestate}.

Окончательное решение принимает человек, которого принято называть лицом, принимающим решение~(ЛПР)~\cite{khmelevski2024realestate}. 

В общем виде постановка задачи МКО может быть представлена в виде \(\{D, f_1, \ldots, f_m\}\), где \(D\)~---~множество допустимых исходов, \(f_i\)~---~числовая функция, заданная на \(D\); при этом \(f_i(a)\) есть оценка исхода \(a \in D\) по \(i\)-му критерию~(\(i = 1, \ldots, m\))~\cite{khmelevski2024realestate}. Такая модель соответствует задаче принятия решений, в которой множество альтернатив соответствует множеству допустимых исходов, а оценочная структура задаётся вектором \((f_1, \ldots, f_m)\). Критерий \(f_i\) называется позитивным, если ЛПР стремится к его увеличению, и негативным, если ЛПР стремится к его уменьшению.

В задачах минимизации решение \(x_1\) доминирует над решением \(x_2\) (\(x_1~\prec~x_2\)), в случаях~\cite{polkovnikova2024evo}: 
\begin{enumerate}
	\item \(x_1\) не хуже \(x_2\) во всех целевых функциях;
	\item \(x_1\) строго лучше, чем \(x_2\) по крайней мере по одной целевой функции.
\end{enumerate}

Множеством Парето называется множество всех допустимых решений задачи многокритериальной оптимизации, которые не доминируются ни одним другим допустимым решением.

Фронт Парето~---~это множество всех векторных значений критериев, соответствующих решениям из множества Парето.

\subsection{Задачи поиска путей в графе}

Граф \(G\) состоит из двух множеств~---~множества вершин и множества рёбер, причём для каждого ребра указана пара вершин, которые это ребро соединяют. Вершины и рёбра называются элементами графа. 

Ориентированный граф \(G\) задаётся двумя множествами
\( G = (V, E), \)
где \(V\)~---~конечное множество, элементы которого называют вершинами (узлами); \(E\)~---~множество упорядоченных пар на \(V\) (подмножество \(V \times V\)), элементы которого называют дугами~\cite{belousov2025discrete}.

Если дуга \(e = (u, v)\), то говорят, что дуга \(e\) ведёт из вершины \(u\) в вершину \(v\), и обозначают это \(u \to v\)~\cite{belousov2025discrete}.

Путь в ориентированном графе \(G\)~---~это последовательность вершин~(конечная или бесконечная) \(
v_0, v_1, \ldots, v_n, \ldots\) такая, что \(v_i \to v_{i+1}\) для любого \(i\), если \(v_{i+1}\) существует~\cite{belousov2025discrete}.

Простой путь~---~это путь, все вершины которого, кроме, быть может, первой и последней, попарно различны~\cite{belousov2025discrete}.

Взвешенный граф~---~это граф, в котором каждому ребру присвоено числовое значение, называемое весом.

Однокритериальная задача поиска пути заключается в нахождении пути между двумя заданными вершинами взвешенного графа, при котором оптимизируется один скалярный критерий, обычно равный сумме весов рёбер на пути.

Векторная стоимость ребра~---~это обобщение понятия веса ребра во взвешенном графе, при котором каждому ребру сопоставляется вектор числовых значений, отражающих несколько критериев оценки (например, время, стоимость, надёжность). Каждому ребру
\(
v \in E
\)
ставится в соответствие вектор
\[
w(v) = \bigl(w_1(v), w_2(v), \ldots, w_m(v)\bigr),
\]
где каждая компонента соответствует отдельному критерию оптимальности.

Стоимость пути в пространстве критериев~---~это вектор, получаемый агрегированием векторных стоимостей всех рёбер, входящих в данный путь, и отражающий значения каждого критерия для всего пути в целом. Для пути
\[
P = \langle v_1, v_2, \ldots, v_k \rangle
\]
его стоимость определяется как
\[
W(P) = \sum_{i=1}^{k} w(v_i),
\]
где суммирование выполняется покомпонентно. Полученный вектор используется для сравнения путей на основе отношения доминирования или других правил многокритериального сравнения.

Нормализация и масштабирование критериев~---~это процедуры преобразования значений критериев к сопоставимому числовому масштабу, применяемые в многокритериальных задачах для устранения различий в размерностях и диапазонах значений. Нормализация обычно сводит значения критериев к безразмерному интервалу
\(
[0,1],
\)
тогда как масштабирование изменяет диапазон значений с сохранением относительных пропорций. 
