\chapter*{\hfill{\centering ВВЕДЕНИЕ}\hfill}

\addcontentsline{toc}{chapter}{ВВЕДЕНИЕ}

В задачах городской логистики может возникнуть необходимость поиска маршрута по нескольким критериям. Например, при мультимодальных поездках оптимизируют время с учётом расписаний и пересадок~\cite{teslya2016multimodal}, а при проектировании транспортной сети сопоставляют стоимость, время и надёжность инфраструктуры~\cite{medvedeva2024rail}. 

Целью данной работы является исследование основных методов мультикритериального поиска путей в графах.

Для достижения поставленной цели необходимо выполнить следующие задачи:
\begin{enumerate}[label=---]
	\item проанализировать предметную область логистики и транспортных сетей;
	\item формализовать задачу мультикритериального поиска путей в графах в виде IDEF0 диаграммы;
	\item рассмотреть основные методы или группы методов решения задачи мультикритериального поиска путей;
	\item провести сравнительный анализ рассмотренных методов.
\end{enumerate}
