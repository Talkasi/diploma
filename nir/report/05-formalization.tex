\chapter{Формализация задачи}

Задача мультикритериального поиска путей в графе формально определяется как поиск множества Парето-оптимальных путей в ориентированном графе \(G(V,E)\) от стартовой вершины \(n_S \in V \) до конечной \(n_{End} \in V \), где каждый путь \(p_i=(n_S, \ldots, n_{End})\) имеет переменную длину \(K\) и оценивается по набору целевых функций \(\vec{f}(p)\), подлежащих минимизации~\cite{MARISTANYDELASCASAS2021105424}. Эта задача является обобщением классической задачи поиска кратчайшего пути и также известна в англоязычной литературе как Multi-Objective Shortest Path Problem~(MOSP Problem).

Следует отметить, что в зависимости от класса применяемых методов под решением
многокритериальной задачи поиска путей может пониматься либо точное построение
множества Парето-оптимальных путей, либо получение его аппроксимации. Точные методы
ориентированы на полное перечисление множества Парето, тогда как эвристические и
популяционные методы позволяют получить приближённое представление фронта Парето
при приемлемых вычислительных затратах.

Под компромиссным путём далее будет пониматься путь, который либо является Парето-оптимальным,
либо принадлежит аппроксимации множества Парето и не доминируется другими
допустимыми путями в рамках принятой точности аппроксимации.

Ниже представлена IDEF0-диаграмма решения задачи мультикритериального поиска путей в графе.
\begin{figure}[h!]
	\centering
	\includegraphics[width=0.7\textwidth]{inc/img/idef0.pdf}
	\caption{IDEF0-диаграмма решения задачи мультикритериального поиска путей в графе}
	\label{fig:idef0-task}
\end{figure}
