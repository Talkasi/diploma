\chapter{Сравнение основных методов}

Для сравнения групп методов, рассмотренных ранее, выделены следующие критерии:

\begin{enumerate}
    \item Доступность: способность метода корректно находить Парето-оптимальные пути в невыпуклых областях фронта Парето.
    \item Точность: способность получать точные Парето-оптимальные решения.
    \item Полнота: способность метода получать полное множество Парето-оптимальных путей.
    \item Вычислительная сложность.
\end{enumerate}

\begin{table}[h]
\centering
\caption{Сравнение групп методов решения многокритериальной задачи поиска путей}
\label{tab:comparison_methods}


\begin{tabular}{|m{3.1cm}|>{\centering\arraybackslash}m{3.3cm}|
>{\centering\arraybackslash}m{2.3cm}|
>{\centering\arraybackslash}m{2.1cm}|
>{\centering\arraybackslash}m{4cm}|}

\hline
\textbf{Методы} 
& \textbf{Доступность} 
& \textbf{Точность} 
& \textbf{Полнота}
& \textbf{Вычислительная сложность} \\ \hline

\makecell[l]{Методы\\скаляризации}
& $+/-$
& $-$
& $-$
& \makecell{Полиномиальная\\\cite{article}} \\ \hline

\makecell[l]{Точные\\методы}
& $+$ 
& $+$ 
& $+$
& \makecell{Экспоненциальная\\\cite{euler2024}} \\ \hline

\makecell[l]{Метаэвристи-\\ческие\\методы}
& $+$ 
& $-$ 
& $-$
& \makecell{Полиномиальная\\\cite{May2023ACOReport, BianQian2022NSGAII}} \\ \hline

\end{tabular}
\end{table}

На основе таблицы можно сформулировать следующие выводы:

\begin{enumerate}[label=---]
    \item Методы скаляризации характеризуются низкой вычислительной сложностью и обеспечивают высокую доступность решений при выпуклом фронте Парето, однако не позволяют корректно восстанавливать невыпуклые области, не обеспечивают точности в таких случаях и не гарантируют полноты множества Парето-оптимальных путей.
    
    \item Точные методы обладают максимальной доступностью, обеспечивают получение полностью корректного и полного множества Парето-оптимальных путей с высокой точностью, но их практическое применение существенно ограничено экспоненциальной вычислительной сложностью, обусловленной ростом количества недоминируемых меток.
    
    \item Метаэвристические методы демонстрируют высокую доступность и способны находить решения в невыпуклых областях фронта Парето, однако предоставляют только приближённые результаты, не обеспечивают точности и не гарантируют полноты множества Парето-оптимальных путей, несмотря на полиномиальную вычислительную сложность.
\end{enumerate}
